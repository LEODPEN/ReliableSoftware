\documentclass[11pt,a4paper,fleqn]{article}
\usepackage[utf8]{inputenc}
\usepackage{bsymb}
%usepackage[space]{ctex}
\usepackage{natbib}
\usepackage{graphicx}
\usepackage{algorithmic_pf}
\usepackage{indentfirst}
\usepackage[export]{adjustbox}
\usepackage{etoolbox}
\expandafter\patchcmd\csname Gin@ii\endcsname
  {\setkeys {Gin}{#1}}
  {%
    \setkeys {Gin}
      {max width=\textwidth,max height=.5\textwidth,keepaspectratio,#1}%
  }
  {}{}
\title{Homework 9}

\author{10175101126,10175101226,10175101117}
\date{May 2019}
\setkeys{Gin}{max width=\linewidth}
\begin{document}
\maketitle


\section{Exercise 1:From PRE and POST, prove that $g$ is equal to  $f^{-1}$}
\noindent
$PRE,\ldots,POST ,f \in 0 \upto n-2 \rightarrow 1 \upto n-1, \forall i \cdot i\in 0\upto n-2 \Rightarrow f(i)=i+1 \vdash g = f^{-1}  $\\
$PRE,\ldots,POST ,f \in 0 \upto n-2 \rightarrow 1 \upto n-1, i\in 0\upto n-2 , i \rightarrow i+1 \in f \vdash g = f^{-1}  $\\
since we know that :\\
$h\in S \rightarrow T,a \rightarrow b \in h \vdash a \in S,b=h(a) $\\
we have:\\
$ f \in 0 \upto n-2 \rightarrow 1 \upto n-1,i\in 0\upto n-2 , i \rightarrow i+1 \in f \vdash i \in 0\upto n-2,i+1 = f(i) $\\
$ i\in 0\upto n-2 ,i+1 = f(i) ,\ldots,POST  \vdash g = f^{-1} \quad \textb{AH} $\\
$ i\in 0\upto n-2 ,i+1 = f(i) ,\ldots, g \in 1 \upto n-1 \rightarrow 0 \upto n-2, \forall i \cdot i \in 1 \upto n-1 \Rightarrow g(i) = i-1  \vdash g = f^{-1}  $\\
since we know :\\
$ h \in S \rightarrow T, a \in S,b = h(a) \vdash a \rightarrow b \in h $\\
we have:\\
$ g \in 1 \upto n-1 \rightarrow 0\upto n-2, i+1 \in 1 \upto n-1, i = g(i+1) \vdash i+1 \rightarrow i \in g $\\
$ i\in 0\upto n-2 ,i+1 = f(i) ,i \rightarrow i+1 \in f,i+1 \rightarrow i \in g  \vdash g = f^{-1}  \quad \textb{AH}$\\
$ i\in 0\upto n-2 ,i+1 \rightarrow i \in f^{-1},i+1 \rightarrow i \in g  \vdash g = f^{-1} $\\
$ i\in 0\upto n-2 ,i+1 \rightarrow i \in f^{-1},i+1 \rightarrow i \in g  \vdash g = f^{-1} $\\
$ i\in 0\upto n-2 , f^{-1} = g  \vdash g = f^{-1} \quad \textb{HYP}$\\
Obvious.


\section{Exercise2 }

\subsection{Propose some invariants and variant}
\noindent
invariants \\
\forall i \in 1 \upto p-1,g(i)=i-1 \\
\forall i \in p \upto n-1,g(i)=i+1 \\
p\leq n\\
\subsection{ Prove these invariants and variant}
\subsubsection{prove the first invariant }
\noindent
PRE \vdash [g:=( \{0\} \domsub f)\cup \{n-1 \mapsto n\},p:=1]\forall i \in 1 \upto p-1,g(i)=i-1  \;INI \\
PRE \vdash [g:=( \{0\} \domsub f)\cup \{n-1 \mapsto n\}]\forall i \in 1 \upto 0,g(i)=i-1  \\
i\;doesn't\;exist \\
PRE,p \neq n, \forall i \in 1 \upto p-1,g(i)=i-1 \vdash [g(p):=p-1,p:=p+1]\forall i \in 1 \upto p-1,g(i)=i-1 \;INV \\
PRE,p \neq n, \forall i \in 1 \upto p-1,g(i)=i-1 \vdash [g(p):=p-1]\forall i \in 1 \upto p,g(i)=i-1 \\
PRE,p \neq n, \forall i \in 1 \upto p-1,g(i)=i-1 \vdash [g(p):=p-1]\forall i \in 1 \upto p-1 \cup i=p,g(i)=i-1 \\
first\; case:\; i \in 1 \upto p-1 \\
PRE,p \neq n, \forall i \in 1 \upto p-1,g(i)=i-1 \vdash [g(p):=p-1]\forall i \in 1 \upto p-1,g(i)=i-1 \; HYP \\
second\; case:\; i = p \\
PRE,p \neq n, \forall i \in 1 \upto p-1,g(i)=i-1 \vdash [g(p):=p-1]i=p,g(i)=i-1 \\
PRE,p \neq n, \forall i \in 1 \upto p-1,g(i)=i-1 \vdash [g(p):=p-1]g(p)=p-1 \\
\subsubsection{prove the second invariant}
\noindent
PRE \vdash [g:=( \{0\} \domsub f)\cup \{n-1 \mapsto n\},p:=1]\forall i \in p \upto n-1,g(i)=i+1  \;INI \\
PRE \vdash [g:=( \{0\} \domsub f)\cup \{n-1 \mapsto n\}]\forall i \in 1 \upto n-1,g(i)=i+1 \\
it's \; proved \; since \; g \; is \; assigned \; like \; that \\
PRE,p \neq n, \forall i \in p \upto n-1,g(i)=i+1 \vdash [g(p):=p-1,p:=p+1]\forall i \in p \upto n-1,g(i)=i+1 \; INV \\
PRE,p \neq n, \forall i \in p \upto n-1,g(i)=i+1 \vdash [g(p):=p-1]\forall i \in p+1 \upto n-1,g(i)=i+1 \\
PRE,\forall i \in p \upto n-1 \vdash i \in p+1 \upto n-1 \\
PRE,p \neq n, \forall i \in p \upto n-1,g(i)=i+1, \forall i \in p+1 \upto n-1 \vdash [g(p):=p-1]\forall i \in p+1 \upto n-1,g(i)=i+1 \;AH \;HYP \\
\subsubsection{prove the third invariant}
\noindent
$\ldots,n\geq 2\vdash [p:=1]p\leq n\quad\text{INI}$\\
$\ldots,n\geq 2\vdash 1\leq n$\\
Obvious.\\
$\ldots,p\not=n,p\leq n\vdash [p:=p+1]p\leq n\quad\text{INV}$\\
$\ldots,p\not=n,p\leq n\vdash p+1\leq n$\\
Obvious.\\
\subsubsection{Prove variant}
\noindent
we choose n-p as the variant.\\
$\text{NAT:}$\\
$ \ldots,p\leq n\vdash n-p\in\mathbb{N}$\\
$\text{VAR:}$\\
$ \ldots\vdash [g(p)=p-1;p:=p+1]n-p<n-p$\\
$ \ldots\vdash [g(p)=p-1][p:=p+1]n-p<n-p$\\
$ \ldots\vdash [p:=p+1]n-p<n-p$\\
$ \ldots\vdash n-(p+1)<n-p$\\
$ \ldots\vdash -1<0$\\
Obvious.


\subsection{Prove POST}
\noindent
$PRE,p=n,\forall i\cdot i\in 1\upto p-1\Rightarrow g(i)=i-1$\\
$\vdash g\in 1\upto n-1\rightarrow 0\upto n-2,
\forall i\cdot i\in 1\upto n-1\Rightarrow g(i)=i-1$\\
$PRE,p=n,\forall i\cdot i\in 1\upto p-1\Rightarrow g(i)=i-1$\\
$\vdash g\in 1\upto n-1\rightarrow 0\upto n-2\quad\text{SPLIT}$\\
Obvious, we have prove it in exercise 1.\\
$PRE,p=n,\forall i\cdot i\in 1\upto p-1\Rightarrow g(i)=i-1$\\
$\vdash\forall i\cdot i\in 1\upto n-1\Rightarrow g(i)=i-1\quad\text{SPLIT}$\\
$PRE,\forall i\cdot i\in 1\upto n-1\Rightarrow g(i)=i-1$\\
$\vdash\forall i\cdot i\in 1\upto n-1\Rightarrow g(i)=i-1\quad\text{EQL\quad and\quad HYP}$\\


%\bibliographystyle{plain}
%\bibliography{references}
\end{document}
